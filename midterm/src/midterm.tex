\documentclass{article}[11pt]
\usepackage{amsmath}


\usepackage{geometry}
 \geometry{
 a4paper,
 total={170mm,257mm},
 left=30mm,
 top=30mm,
 right=30mm,
 bottom=30mm,
 }



\title{Optimising Robotic Swarm Movement}
\author{Manish Kumar Bera \\
Mentor: Dr. Indranil Saha
}

%\topmargin=7pt

\begin{document}
	\maketitle

	\section{Introduction}
		Swarm robotics is a field of multi-robotics in which large number of robots are coordinated in a centralized$^{[1]}$ or distributed$^{[3]}$ way.
		It is supposed that a desired collective behavior emerges from the interactions between the robots and interactions of robots with the environment.
		With  recent  technological  developments, producing a large number of inexpensive robots that are  equipped  with  sophisticated  sensing,  computation and communication tools has become a reality$^{[5]}$. 
		Swarm robots can be used to form various spatial arrangements for any required purpose.
		Robots need to move simultaneosly, to go from one position to another, to collectively form a new configuration.
		Describing  complex  spatial  specifications  for  swarms  is a  non-trivial  task.

	\section{Objective}
		Describing  complex  spatial  specifications  for  swarms  is a  non-trivial  task.
		Our goal is to optimise the cost of the collective motion of the robots from initial configuration to final configuration.

	\section{Methodology}
		We consider a grid $A$ of size $n \times n$.
		We will refer the collction of all the robots in the grid as \textit{swarm}. 
		We will have $R$ number of robots, such that a robot will occupy a cell of the grid.
		No two robots can occupy the same cell of the grid, and a robot will occupy only a single cell of the grid.
		We will refer to a \textit{move} as the movement of a robot.
		%A robot can move only to an adjacent cell upon executing a \textit{move}.
		We define a \textit{timestep} as the time period after which a robot can execute a \textit{move}.
		A robot can execute a \textit{move} on each \textit{timestep}.
		Each move will be selected from a set of \textit{Motion Primitives}. 
		%A robot may or may not execute a move on a \textit{timestep}.

	%\section{Methodology}
		Let $L$ be the total number of \textit{timesteps} which the \textit{swarm} takes to go from initial configuration to final configuration.  
		Let $x_t^i$ be a variable that describes the position of $i$th robot after $t$ time steps.
		We define \textit{swarm position at timestep $t$},  $X_t$  as follows:
		$$
			X_t = (x_t^1, x_t^2, x_t^3, . . . , x_t^R)
		$$
		i.e., $X_t$ is an ordered collection of tuples, with the $i$th tuple describing the position of the $i$th robot after $t$ \textit{timesteps}. Thus, $X_t$ describes the position of the \textit{swarm} in the grid after $t$ \textit{timesteps}.		
		So, we can say that $X_0 =(x_0^1, x_0^2, x_0^3, . . . , x_0^R)$ represents the initial configuration of the \textit{swarm}, and $X_L = (x_L^1, x_L^2, x_L^3, . . . , x_L^R)$ represents the final configuration of the \textit{swarm}.

		The movement of the \textit{swarm} will obey a set of rules. The progress from one \textit{swarm position} to another will always follow certain conditions. Our aim is to develop a set of constraints, whose solution will give an optimal for the \textit{swarm}.
		We will frame these constraints using a set of boolean variables, which will be encorporated into a formula $\phi$.
		We define a \textit{solution} $S$:
		$$
			S = [X_0, X_1, X_2, . . . , X_L]  
		$$

		where $S$ is a sequence of \textit{swarm positions} which occur in the chronological order of \textit{timesteps}. The symbol $S$ represents the sequence in which the \textit{swarm} movement can occur from initial to final configuration. A \textit{solution} $S$ is said to be valid iff:
		\begin{gather}
				S \models \phi \\
			\Leftrightarrow  [X_0, X_1, X_2, . . . , X_L] \models \phi
		\end{gather}
		The problem will thus be reduced from an optimal path finding problem to a problem of finding satisfiability(SAT)$^{[2]}$. 
		We will use a SAT solver (like the MiniSAT$^{[4]}$) to compute the solution to the above model.

	\section{Work done till now}
		\subsection{boolean variables}
			I define $X(t, k, x, y)$ as a boolean variable s.t. $X(t, k, x, y)$ is \textit{TRUE} iff at time step $t$, the $k$th robot is at position $(x,y)$.
		\subsection{initial and final states}
			I define $(x_0^k,y_0^k)$ to be the initial position of the $k$th robot.
			Similarly, $(x_L^K, y_L^k)$ is the final position of the $k$th robot.\\
			The constraints for initial and final positions will be:
			\begin{equation}
			\begin{split}
				\forall{t}\forall{k}, X(0, k, x, y) &= TRUE;\ if \ x = x_0^k \ and \ y=y_0^k\\
				                                    &= FALSE;\ otherwise
			\end{split}
			\end{equation}
			\begin{equation}
			\begin{split}
				\forall{t}\forall{k}, X(L, k, x, y) &= TRUE;\ if \ x = x_L^k \ and \ y=y_L^k \\
				                                    &= FALSE;\ otherwise
			\end{split}
			\end{equation}			

		\subsection{motion primitives}
			I assumed that the robots will have five simple motion primitives. each robot can:\\
			\textit{
				1. Move up by one cell\\
				2. Move down by one cell\\
				3. Move right by one cell\\
				4. Move left by one cell\\
				5. Stay in that cell\\ 
			}
			Each robot has to perform one of the above mentioned motion primitives at each time step. Whether the robots are allowed to perform a particular primitive at a give timestep also depends on other constraints.\\

			The constraints for motion primitives will be:
			$$
				\forall{t}\forall{k},\ X(t+1, k, x, y) \Rightarrow X(t,k,x,y)\vee X(t,k,x-1,y)\vee X(t,k,x,y-1)\vee X(t,k,x+1,y)\vee X(t,k,x,y+1)
			$$

		\subsection{obstacle avoidance}
			Let the number of obstacles be $G$. Let $(x_{obs}^g, y_{obs}^g)$ be the position of the $g$th obstacle where $g \in \{0, 1, . . . , G-1\}$. Now each robot has to avoid each obstacle at every timestep.\\
			The constraints for obstacle avoidance will be:
			$$
				\forall{t}\forall{k},\ X(t, k, x, y) = FALSE;\ if\ \exists g \in \{0,...,G-1\}\ s.t.\ x=x_{obs}^g\ and\ y=y_{obs}^g
			$$

		\subsection{collision avoidance}
			Each robot must avoid getting hit by another robot. There will be two types of collision.
			\subsubsection{same space collision avoidance}
				This refers to the collision when two robots try to occupy the same cell of the grid. The constraints for this type of collision avoidance will be:\\
				$\forall t \in \{0, 1, ..., L\};\ \forall k_1,k_2 \in \{0, 1, ..., R-1\}\ s.t.\ k_1 \neq k_2;\ \forall x,y \in \{0,1,...,n-1\};$
				$$
					\neg (X(t, k_1, x, y) \wedge X(t, k_2, x, y))
				$$
			\subsection{headon collision avoidance}
				This refers to the collision type where two robots move into each other. The constraints for this type of collision avoidance will be:\\
				$\forall t \in \{0, 1, ..., L\};\ \forall k_1,k_2 \in \{0, 1, ..., R-1\}\ s.t.\ k_1 \neq k_2;\ \forall x,y \in \{0,1,...,n-1\};$
				$$
					\neg((X(t, k_1, x, y) \wedge X(t, k_1, x+1, y)) \wedge (X(t, k_2, x+1, y) \wedge X(t, k_2, x, y))) \wedge
				$$
				$$
					\neg((X(t, k_1, x, y) \wedge X(t, k_1, x, y+1)) \wedge (X(t, k_2, x, y+1) \wedge X(t, k_2, x, y)))
				$$


		\subsection{mark1: The first prototype}
			I used the MiniSAT, a general purpose SAT solver for solving SAT problem instances. 



		\subsection{mark2}
		\subsection{mark3}
		\subsection{mark3.1}
		\subsection{mark3.2}

		\begin{thebibliography}{9}
			\bibitem{CalinBelta} 
			Iman Haghighi, Sadra Sadraddini, and Calin Belta. 
			\textit{Robotic  Swarm  Control  from  Spatio-Temporal  Specifications}.
			2016 IEEE 55th Conference on Decision and Control (CDC), ARIA Resort \& Casino, December 12-14, 2016, Las Vegas, USA. 			
			
			\bibitem{Wikipedia} 
			Wikipedia
			\textit{Boolean satisfiability problem}. \\
			\texttt{https://en.wikipedia.org/wiki/Boolean\_satisfiability\_problem} 
			
			\bibitem{harvard}			
			\author{Caroline Perry.}
			\textit{A self-organizing thousand-robot swarm.}
			\date{August 14, 2014.} \\
			\texttt{https://www.seas.harvard.edu/news/2014/08/self-organizing-thousand-robot-swarm}
			
			\bibitem{miniSAT} 
			\textit{The MiniSAT Page.}
			\author{Niklas Eén, Niklas Sörensson} \\
			%\date{August 14, 2014}
			\texttt{http://minisat.se/Main.html}

			\bibitem{kilobot}			
			\author{Self Organizing Systems Research Group.}
			\textit{The Kilobot Project.}
			%\date{August 14, 2014.} \\
			\texttt{https://www.eecs.harvard.edu/ssr/projects/progSA/kilobot.html}
		\end{thebibliography}
\end{document}
