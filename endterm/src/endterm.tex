\documentclass{article}[11pt]
\usepackage{amsmath}
%\usepackage[letterpaper, portrait, margin=0.5in]{geometry}
%\usepackage[margin=0.5in]{geometry}

\usepackage{geometry}
 \geometry{
 a4paper,
 total={170mm,257mm},
 left=25mm,
 top=25mm,
 right=25mm,
 bottom=25mm,
 }



\title{Optimising Robotic Swarm Movement}
\author{Manish Kumar Bera \\
Mentor: Dr. Indranil Saha
}

%\topmargin=7pt

\begin{document}
	\maketitle

	\section{Objective}
		Describing  complex  spatial  specifications  for  swarms  is a  non-trivial  task.
		Our goal is to optimise the cost of the collective motion of the robots from initial configuration to final configuration.

	\section{Methodology}
		We consider a grid $A$ of size $n \times n$.
		We will refer the collection of all the robots in the grid as \textit{swarm}. 
		We will have $R$ number of robots.% such that a robot will occupy a cell of the grid.
		No two robots can occupy the same cell of the grid, and a robot will occupy only a single cell of the grid.
		In a timestep, each robot can either move to an adjacent cell, or may remain in the same cell.
		%We will refer to a \textit{move} as the movement of a robot.
		%A robot can move only to an adjacent cell upon executing a \textit{move}.
		We define a \textit{timestep} as the time period after which a robot can execute a \textit{move}.
		%A robot can execute a \textit{move} on each \textit{timestep}.
		%Each move will be selected from a set of \textit{Motion Primitives}. 
		%A robot may or may not execute a move on a \textit{timestep}.

	%\section{Methodology}
		Let $L$ be the total number of \textit{timesteps} which the \textit{swarm} takes to go from initial configuration to final configuration.  
		Let $x_t^i$ be a variable that describes the position of $i$th robot after $t$ time steps.
		We define \textit{swarm position at timestep $t$}, $X_t$ as :
		$$
			X_t = (x_t^1, x_t^2, x_t^3, . . . , x_t^R)
		$$
		i.e., $X_t$ is an ordered collection of tuples, with the $i$th tuple describing the position of the $i$th robot after $t$ \textit{timesteps}.
		Thus, $X_t$ describes the position of the \textit{swarm} in the grid after $t$ \textit{timesteps}. So, we can say that $X_0 =(x_0^1, x_0^2, x_0^3, . . . , x_0^R)$ represents the initial configuration of the \textit{swarm}, and $X_L = (x_L^1, x_L^2, x_L^3, . . . , x_L^R)$ represents the final configuration of the \textit{swarm}.

		%The movement of the \textit{swarm} will obey a set of rules. The progress from one \textit{swarm position} to another will always follow certain conditions.
		Our aim is to develop a set of constraints, whose solution will give an optimal solution for the \textit{swarm}.
		We will frame these constraints using a set of boolean variables, which will be encorporated into a formula $\phi$.
		We define a \textit{solution} $S$:
		$$
			S = [X_0, X_1, X_2, . . . , X_L]  
		$$

		where $S$ is a sequence of \textit{swarm positions}.% which occur in the chronological order of \textit{timesteps}.
		%The symbol $S$ represents the sequence in which the \textit{swarm} movement can occur from initial to final configuration.
		A \textit{solution} $S$ is said to be valid iff:
		\begin{gather}
				S \models \phi \\
			\Leftrightarrow  [X_0, X_1, X_2, . . . , X_L] \models \phi
		\end{gather}
		The problem will thus be reduced from an optimal path finding problem to a problem of finding satisfiability(SAT)$^{[2]}$. 
		We will use a SAT solver (like the MiniSAT$^{[4]}$) to compute the solution to the above model.

	\section{Work done till now}
		%\subsection{boolean variables}
		\textbf{defining boolean variables:}\\
			I define $X(t, k, x, y)$ as a boolean variable s.t. $X(t, k, x, y)$ is \textit{TRUE} iff at time step $t$, the $k$th robot is at position $(x,y)$.\\
		
		\textbf{constraints for initial and final states:}\\
			I define $(x_0^k,y_0^k)$ to be the initial position of the $k$th robot.
			Similarly, $(x_L^K, y_L^k)$ is the final position of the $k$th robot.\\
			The constraints for initial and final positions will be:
			\begin{equation}
			\begin{split}
				\forall{t \in \{0,...,L\}}\forall{k \in \{0,...,R-1\}}, X(0, k, x, y) &= TRUE;\ if \ x = x_0^k \ and \ y=y_0^k\\
				                                    &= FALSE;\ otherwise
			\end{split}
			\end{equation}
			\begin{equation}
			\begin{split}
				\forall{t \in \{0,...,L\}}\forall{k \in \{0,...,R-1\}}, X(L, k, x, y) &= TRUE;\ if \ x = x_L^k \ and \ y=y_L^k \\
				                                    &= FALSE;\ otherwise
			\end{split}
			\end{equation}			

		\textbf{constraints for motion primitives:}\\
			I assume that the robots will have five simple motion primitives. each robot can:\\
			\textit{
				1. Move up by one cell\\
				2. Move down by one cell\\
				3. Move right by one cell\\
				4. Move left by one cell\\
				5. Stay in that cell\\ 
			}
			Each robot has to perform one of the above mentioned motion primitives at each time step. Whether the robots are allowed to perform a particular primitive at a give timestep also depends on other constraints.\\

			The constraints for motion primitives will be:
			$$
				\forall{t \in \{0,...,L\}}\forall{k \in \{0,...,R-1\}},\ X(t+1, k, x, y) \Rightarrow X(t,k,x,y)\vee X(t,k,x-1,y)\vee X(t,k,x,y-1)\vee X(t,k,x+1,y)\vee X(t,k,x,y+1)
			$$

		\textbf{constraints for obstacle avoidance:}\\
			Let the number of obstacles be $G$. Let $(x_{obs}^g, y_{obs}^g)$ be the position of the $g$th obstacle where $g \in \{0, 1, . . . , G-1\}$. Each robot has to avoid each obstacle at every timestep.\\
			The constraints for obstacle avoidance will be:
			$$
				\forall{t \in \{0,...,L\}}\forall{k \in \{0,...,R-1\}},\ X(t, k, x, y) = FALSE;\ if\ \exists g \in \{0,...,G-1\}\ s.t.\ x=x_{obs}^g\ and\ y=y_{obs}^g
			$$

		\textbf{constraints for collision avoidance:}\\
			Each robot must avoid getting hit by another robot. There will be two types of collision.\\
			\textit{	(i) constraints for same space collision avoidance:}\\
				This refers to the collision when two robots try to occupy the same cell of the grid. The constraints for this type of collision avoidance will be:\\
				$\forall t \in \{0, 1, ..., L\};\ \forall k_1,k_2 \in \{0, 1, ..., R-1\}\ s.t.\ k_1 \neq k_2;\ \forall x,y \in \{0,1,...,n-1\};$
				$$
					\neg (X(t, k_1, x, y) \wedge X(t, k_2, x, y))
				$$
			\textit{	(ii) headon collision avoidance:}\\
				This refers to the collision type where two robots move into each other. The constraints for this type of collision avoidance will be:\\
				$\forall t \in \{0, 1, ..., L\};\ \forall k_1,k_2 \in \{0, 1, ..., R-1\}\ s.t.\ k_1 \neq k_2;\ \forall x,y \in \{0,1,...,n-1\};$
				$$
					\neg((X(t, k_1, x, y) \wedge X(t, k_1, x+1, y)) \wedge (X(t, k_2, x+1, y) \wedge X(t, k_2, x, y))) \wedge
				$$
				$$
					\neg((X(t, k_1, x, y) \wedge X(t, k_1, x, y+1)) \wedge (X(t, k_2, x, y+1) \wedge X(t, k_2, x, y)))
				$$

		\textbf{IMPLEMENTATION:}\\
			For siolving the constraints, I transform the constraints into CNF form and feed it as input to the MiniSAT tool. For getting the correct value of $L$, we have to iterate over successive integers, starting from 1. For optimisation, I make exponential jumps, i.e., I iterate over 1, 2, 4, 8, ... . After I get satisfiability, say at $2^a$, I do a binary search between $2^{a-1}$ and $2^a$to get the least value of $L$ for which the problem can be solved. I have also used other optimizaton techniques like reusing the constraints calculated for previous $L$ values, instead of calculating them again.

	\section{Further Work:}
		We wish to extend the developed implementation to include more complex motion primitives for smoother trajectories. We also wish to optimise the motion primitive cost using MAX-SAT. We further hope to enhance the compuation speed using techniques like multi-threading, and optimization methods such as pre-computation of constraints.
		\end{document}
